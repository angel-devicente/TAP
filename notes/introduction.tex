These are the notes to be followed in the course ``Programming Techniques''
(``Técnicas de Programación'') of the Master in Astrophysics (ULL).

The objective of the course is to learn some programming techniques necessary in
many scientific codes. In particular, we will study about dynamic data
structures and MPI parallel programming with Fortran.

All the material taught in the course will be motivated by a sample N-body
problem. We will start by developing a naïve and serial code; then we will study
about the much more efficient Barnes-Hut algorithm and in order to implement
this algorithm we will have to learn about dynamic data structures, recursion,
trees, lists, etc. Once a Barnes-Hut serial implementation is finished, we will
focus on learning the basics of parallel programming, in this case using the MPI
library, and on parallelizing the Barnes-Hut version of our code.

We will also touch briefly on code debugging and profiling (both in serial and
in parallel codes) and on other parallel programming models (OpenMP, CUDA,
OpenACC) and how to use them together with MPI. 


