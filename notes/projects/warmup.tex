We start with some simple exercises.

\Level 0 {Hello world}
\commandrefexercise{sec:rank-size}

First of all we need to make sure that you have a working setup for
parallel jobs. The example program \n{helloworld.c} does the
following:
\verbatimsnippet{hello}
Compile this program and run it in parallel. Make sure that the processors
do \emph{not} all say that they are \n{processor 0 out of 1}!

\begin{istc}
  \Level 0 {Trace output}

  We want to make trace files of the parallel runs, for which we'll
  use the TAU utility of the University of Oregon. 
  (For documentation, go to \url{http://www.cs.uoregon.edu/Research/tau/docs.php}.)
  Here are the steps:
  \begin{itemize}
  \item Load two modules:
\begin{verbatim}
module load tau
module load jdk64
\end{verbatim}
  \item Recompile your program with \n{make yourprog}. You'll notice a
    lot more output: that is the TAU preprocessor.
  \item Now run your program, setting environment variables
    \n{TAU_TRACE} and \n{TAU_PROFILE} to~1, and \n{TRACEDIR} and
    \n{PROFILEDIR} to where you want the output to be.  Big shortcut:
    do 
\begin{verbatim}
make submit EXECUTABLE=yourprog
\end{verbatim}
    for a batch job or 
\begin{verbatim}
make idevrun EXECUTABLE=yourprog
\end{verbatim}
    for an interactive parallel
    run. These last two set all variables for you. See if you can find
    where the output went\ldots
  \item Now you need to postprocess the TAU output. Do \n{make tau
    EXECUTABLE=yourprog} and you'll get a file
    \n{taulog_yourprog.slog2} which you can view with the \n{jumpshot}
    program.
  \end{itemize}
\end{istc}

\Level 0 {Collectives}

It is a good idea to be able to collect statistics, so before we do
anything interesting, we will look at MPI collectives;
section~\ref{sec:collective}.

Take a look at \n{time_max.cxx}. This program sleeps for a random
number of seconds: 
\verbatimsnippet{makejitter}
and measures how long the sleep actually was:
\verbatimsnippet{measurejitter}
In the code, this quantity is called `jitter', which is a term
for random deviations in a system.

\begin{exercise}
  Change this program to compute the average jitter by changing the reduction
  operator.
\end{exercise}

\begin{exercise}
  Now compute the standard deviation
  \[ \sigma = \sqrt{\frac{ \sum_i (x_i-m)^2 }{n} } \]
  where $m$ is the average value you computed in the previous exercise.
  \begin{itemize}
  \item Solve this exercise twice: once by following the reduce by a
    broadcast operation and once by using an \n{Allreduce}.
  \item Run your code both on a single cluster node and on multiple
    nodes, and inspect the TAU trace. Some MPI implementations are
    optimized for shared memory, so the trace on a single node may not
    look as expected.
  \item Can you see from the trace how the allreduce is implemented?
  \end{itemize}
\end{exercise}

\begin{exercise}
  Finally, use a gather call to collect all the values on processor
  zero, and print them out. Is there any process that behaves very
  differently from the others?
\end{exercise}

\begin{istc}
For each exercise, submit code, a TAU trace, and an analysis of what
you see in the traces. Submit your work by leaving a code, graphics,
and a writeup in your repository.
\end{istc}

\Level 0 {Linear arrays of processors}

In this section you are going to write a number of variations on
a very simple operation: 
all processors pass a data item to the processor with the next higher
number.
\begin{itemize}
\item In the file \n{linear-serial.c} you will find an implementation
  using blocking send and receive calls.
\item You will change this code to use non-blocking sends and
  receives; they require an \n{MPI_Wait} call to finalize them.
\item Next, you will use \n{MPI_Sendrecv} to arrive at a synchronous,
  but deadlock-free implementation.
\item Finally, you will use two different one-sided scenarios.
\end{itemize}
In the reference code \n{linear-serial.c}, each process defines two buffers:
\verbatimsnippet{linearnumbers}
where \n{other_number} is the location where the data from the left neighbour is going to be stored.

To check the correctness of the program, there is a gather operation on processor zero:
\verbatimsnippet{serialgathercheck}

\Level 1 {Coding with blocking calls}

Passing data to a neighbouring processor should be a very parallel operation.
However, if we code this naively, with \n{MPI_Send} and \n{MPI_Recv},
we get an unexpected serial behaviour, as was explained in section~\ref{sec:blocking}.
\verbatimsnippet{linear}
(Note that this uses an \n{Ssend}; see section~\ref{sec:handshake}
for the explanation why.)

\begin{exercise}
  Compile and run this code, and generate a TAU trace file. Confirm
  that the execution is serial. Does replacing the \n{Ssend} by \n{Send}
  change this?
\end{exercise}
Let's clean up the code a little.
\begin{exercise}
  First write this code more elegantly by using \n{MPI_PROC_NULL}.
\end{exercise}

\Level 1 {A better blocking solution}

The easiest way to prevent the serialization problem of the previous
exercises is to use the \indexmpishow{MPI_Sendrecv} call. This routine
acknowledges that often a processor will have a receive call whenever there
is a send. For border cases where a send or receive is unmatched you can
use \indexmpishow{MPI_PROC_NULL}.

\begin{exercise}
  Rewrite the code using \n{MPI_Sendrecv}. Confirm with a TAU trace
  that execution is no longer serial.
\end{exercise}

Note that the \n{Sendrecv} call itself is still blocking, but
at least the ordering of its constituent send and recv are
no longer ordered in time.

\Level 1 {Non-blocking calls}

The other way around the blocking behaviour is to use \n{Isend} and \n{Irecv}
calls, which do not block. Of course, now you need a guarantee that these 
send and receive actions are concluded; in this case, use \n{MPI_Waitall}.
\begin{exercise}
  Implement a fully parallel version by using \n{MPI_Isend} and
  \n{MPI_Irecv}.
\end{exercise}

\Level 1 {One-sided communication}

Another way to have non-blocking behaviour is to use one-sided
communication.  During a \n{Put} or \n{Get} operation, execution will
only block while the data is being transferred out of or into the
origin process, but it is not blocked by the target. Again, you need a
guarantee that the transfer is concluded; here use
\indexmpishow{MPI_Win_fence}.

\begin{exercise}
  Write two versions of the code: one using \n{MPI_Put} and one with \n{MPI_Get}.
  Make TAU traces.
\end{exercise}
Investigate blocking behaviour through TAU visualizations. 
\begin{exercise}
  If you transfer a large amount of data, and the target processor is
  occupied, can you see any effect on the origin? Are the fences
  synchronized?
\end{exercise}
