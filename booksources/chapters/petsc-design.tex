% -*- latex -*-
%%%%%%%%%%%%%%%%%%%%%%%%%%%%%%%%%%%%%%%%%%%%%%%%%%%%%%%%%%%%%%%%
%%%%%%%%%%%%%%%%%%%%%%%%%%%%%%%%%%%%%%%%%%%%%%%%%%%%%%%%%%%%%%%%
%%%%
%%%% This text file is part of the source of 
%%%% `Parallel Programming in MPI and OpenMP'
%%%% by Victor Eijkhout, copyright 2012-8
%%%%
%%%% petsc.tex : petsc stuff
%%%%
%%%%%%%%%%%%%%%%%%%%%%%%%%%%%%%%%%%%%%%%%%%%%%%%%%%%%%%%%%%%%%%%
%%%%%%%%%%%%%%%%%%%%%%%%%%%%%%%%%%%%%%%%%%%%%%%%%%%%%%%%%%%%%%%%

\Level 0 {Language support}

PETSc is implemented in C, so there is a natural interface
to~C. A~\emph{Fortran90}\index{Fortran90!PETSc interface}
interface exists. The \emph{Fortran77}\index{Fortran77!PETSc interface}
interface is only of
interest for historical reasons.

A \emph{python}\index{Python!PETSc interface} interface was written by
Lisandro Dalcin, and requires separate installation, based on already
defined \indextermtt{PETSC_DIR} and \indextermtt{PETSC_ARCH}
variables.  This can be downloaded at
\url{https://bitbucket.org/petsc/petsc4py/src/master/}, with
documentation at
\url{https://www.mcs.anl.gov/petsc/petsc4py-current/docs/}.

\Level 0 {Startup}
\label{sec:petscinit}

\begin{verbatim}
ierr = PetscInitialize(&Argc,&Args,PETSC_NULL,PETSC_NULL);
MPI_Comm comm = PETSC_COMM_WORLD;

MPI_Comm_rank(comm,&mytid);
MPI_Comm_size(comm,&ntids);
\end{verbatim}

\begin{pythonnote}
  The following works if you don't need commandline options.
\begin{verbatim}
from petsc4py import PETSc
\end{verbatim}
To pass commandline arguments to PETSc, do:
\begin{verbatim}
import sys
from petsc4py import init
init(sys.argv)
from petsc4py import PETSc
\end{verbatim}
\end{pythonnote}

\begin{verbatim}
comm = PETSc.COMM_WORLD
nprocs = comm.getSize(self) 
procno = comm.getRank(self)
\end{verbatim}

\Level 0 {Commandline options}

See section~\ref{sec:petscinit} about passing the commandline options.

\begin{pythonnote}
  In Python, do not specify the initial hyphen of an option name.
\begin{verbatim}
hasn = PETSc.Options().hasName("n")
\end{verbatim}
\end{pythonnote}

\Level 0 {Printing}

Printing screen output in parallel is tricky. If two processes execute
a print statement at more or less the same time there is no guarantee
as to in what order they may appear on screen. (Even attempts to have
them print one after the other may not result in the right ordering.)
Furthermore, lines from multi-line print actions on two processes may
wind up on the screen interleaved.

PETSc has two routines that fix this problem. First of all, often the
information printed is the same on all processes, so it is enough if
only one process, for instance process~0, prints it.
%
\petscRoutineRef{PetscPrintf}

If all processes need to print, there is a routine that forces the
output to appear in process order.
%
\petscRoutineRef{PetscSynchronizedPrintf}

To make sure that output is properly flushed from all system buffers
use a flush routine:
%
\petscRoutineRef{PetscSynchronizedFlush}
%
where for ordinary screen output you would use \n{stdout} for the file.

\begin{pythonnote}
  Since the print routines use the python \n{print} call, they
  automatically include the trailing newline. You don't have to
  specify it as in the C~calls.
\end{pythonnote}

