% -*- latex -*-
%%%%%%%%%%%%%%%%%%%%%%%%%%%%%%%%%%%%%%%%%%%%%%%%%%%%%%%%%%%%%%%%
%%%%%%%%%%%%%%%%%%%%%%%%%%%%%%%%%%%%%%%%%%%%%%%%%%%%%%%%%%%%%%%%
%%%%
%%%% This text file is part of the source of 
%%%% `Parallel Programming in MPI and OpenMP'
%%%% by Victor Eijkhout, copyright 2012-8
%%%%
%%%% mpi-shared.tex : about shared memory in MPI
%%%%
%%%%%%%%%%%%%%%%%%%%%%%%%%%%%%%%%%%%%%%%%%%%%%%%%%%%%%%%%%%%%%%%
%%%%%%%%%%%%%%%%%%%%%%%%%%%%%%%%%%%%%%%%%%%%%%%%%%%%%%%%%%%%%%%%

The one-sided MPI calls (chapter~\ref{ch:mpi-onesided}) can be used to
emulate shared memory. In this chapter we will look at the ways MPI
can interact with the presence of actual shared memory. Many MPI
implementations have optimizations that detect shared memory and can
exploit it, but that is not exposed to the programmer. The
\indextermbus{MPI}{3} standard added routines that do give the programmer
that knowledge.

\Level 0 {Recognizing shared memory}
\label{mpi-comm-split-type}

MPI's one-sided routines take a very symmetric view of processes:
each process can access the window of every other process (within a communicator).
Of course, in practice there will be a difference in performance
depending on whether the origin and target are actually
on the same shared memory, or whether they can only communicate through the network.
For this reason MPI makes it easy to group processes by shared memory domains
using \indexmpishow{MPI_Comm_split_type}.

\mpiRoutineRef{MPI_Comm_split_type}

Here the \n{split_type} parameter has to be from the following (short) list:
\begin{itemize}
\item \indexmpishow{MPI_COMM_TYPE_SHARED}: split the communicator into subcommunicators
  of processes sharing a memory area.
\end{itemize}

In the following example, \n{CORES_PER_NODE} is a platform-dependent
constant:
%
\cverbatimsnippet{commsplittype}

\Level 0 {Shared memory for windows}

Processes that exist on the same physical shared memory should be able
to move data by copying, rather than through MPI send/receive calls
--~which of course will do a copy operation under the hood.
In order to do such user-level copying:
\begin{enumerate}
\item We need to create a shared memory area with
  \indexmpishow{MPI_Win_allocate_shared}, and
\item We need to get pointers to where a process' area is in this
  shared space; this is done with \indexmpishow{MPI_Win_shared_query}.
\end{enumerate}

\Level 1 {Pointers to a shared window}

The first step is to create a window (in the sense of one-sided MPI;
section~\ref{sec:windows}) on the processes on one node.
Using the \indexmpishow{MPI_Win_allocate_shared} call presumably will
put the memory close to the 
\indexterm{socket} on which the process runs.

\cverbatimsnippet[code/mpi/shared.c]{mpisharedwindow}

\mpiRoutineRef{MPI_Win_allocate_shared}

The memory allocated by \indexmpishow{MPI_Win_allocate_shared} is
contiguous between the processes. This makes it possible to do address
calculation. However, if a cluster node has a \ac{NUMA} structure, for
instance if two sockets have memory directly attached to each, this
would increase latency for some processes. To prevent this, the key
\indexmpishow{alloc_shared_noncontig} can be set to \n{true} in the
\indexmpishow{MPI_Info} object.

\Level 1 {Querying the shared structure}

Even though the window created above is shared, that doesn't mean it's
contiguous. Hence it is necessary to retrieve the pointer to the area
of each process that you want to communicate with.

\cverbatimsnippet[code/mpi/shared.c]{mpisharedpointer}

\mpiRoutineRef{MPI_Win_shared_query}

\Level 1 {Heat equation example}

As an example, which consider the 1D heat equation. On each process we
create a local area of three point:
%
\cverbatimsnippet{allocateshared3pt}

\Level 1 {Shared bulk data}

In applications such as \indexterm{ray tracing}, there is a read-only
large data object (the objects in the scene to be rendered) that is
needed by all processes. In traditional MPI, this would need to be
stored redundantly on each process, which leads to large memory
demands. With MPI shared memory we can store the data object once per
node. Using as above \indexmpishow{MPI_Comm_split_type} to find a
communicator per \ac{NUMA} domain, we store the object on process zero
of this node communicator.

\begin{exercise}
  \label{ex:shareddata}
  Let the `shared' data originate on process zero in
  \indexmpishow{MPI_COMM_WORLD}. Then:
  \begin{itemize}
  \item create a communicator per shared memory domain;
  \item create a communicator for all the processes with number zero on their
    node;
  \item broadcast the shared data to the processes zero on each node.
  \end{itemize}
\end{exercise}
