  \label{ex:rowcolcomm}
  Organize your processes in a grid, and make subcommunicators for
  the rows and columns. For this compute the row and column number of
  each process.

  In the row and column communicator, compute the rank. For instance,
  on a $2\times3$ processor grid you should find:
\begin{verbatim}
Global ranks:  Ranks in row:  Ranks in colum:
  0  1  2      0  1  2        0  0  0
  3  4  5      0  1  2        1  1  1
\end{verbatim}

  Check that the rank in the row communicator is the column number,
  and the other way around.

  Run your code on different number of processes, for instance a
  number of rows and columns that is a power of~2, or that is a prime number.
\begin{tacc}
    This is one occasion where you could use \n{ibrun -np 9};
    normally you would \emph{never} put a processor count on \n{ibrun}.
\end{tacc}
