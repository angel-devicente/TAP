  \label{ex:procname}
  Now use the command \indexmpishow{MPI_Get_processor_name}
  in between the
  init and finalize statement, and print out on what processor your process runs.
  Confirm that you are able to run a program that uses two different nodes.

  The character buffer needs to be allocated by you, it is not
  created by MPI, with size at
  least \indexmpishow{MPI_MAX_PROCESSOR_NAME}.

\begin{tacc}
    TACC nodes have a hostname \n{cRRR-CNN}, where RRR is the rack number, C is the chassis
    number in the rack, and NN is the node number within the chassis. Communication
    is faster inside a rack than between racks!
\end{tacc}
