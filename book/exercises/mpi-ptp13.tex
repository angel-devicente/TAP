  \label{ex:exchangesort}
  A very simple sorting algorithm is \indextermsub{swap}{sort} or
  \indextermsub{odd-even transposition}{sort}:
  pairs of processors compare data, and if necessary exchange. The
  elementary step is called a \indexterm{compare-and-swap}: in a pair
  of processors each sends their data to the other; one keeps the
  minimum values, and the other the maximum.
  For simplicity, in this exercise we give each processor just a single number.

  The exchange sort algorithm is split in even and odd stages, where
  in the even stage, processors $2i$ and $2i+1$ compare and swap data,
  and in the odd stage, processors $2i+1$ and $2i+2$ compare and swap.
  You need to repeat this $P/2$ times, where $P$~is the number of
  processors; see figure~\ref{fig:swapsort1}.

  Implement this algorithm using \indexmpishow{MPI_Sendrecv}. (Use
  \indexmpishow{MPI_PROC_NULL} for the edge cases if needed.)
  Use a gather call to print the global state of the distributed array
  at the beginning and end of the sorting process.
