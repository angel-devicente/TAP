% -*- latex -*-
%%%%%%%%%%%%%%%%%%%%%%%%%%%%%%%%%%%%%%%%%%%%%%%%%%%%%%%%%%%%%%%%
%%%%%%%%%%%%%%%%%%%%%%%%%%%%%%%%%%%%%%%%%%%%%%%%%%%%%%%%%%%%%%%%
%%%%
%%%% This text file is part of the lecture slides for
%%%% `Parallel Programming in MPI and OpenMP'
%%%% by Victor Eijkhout, copyright 2012-8
%%%%
%%%% coursemacs.tex : macros for the lecture slides
%%%%
%%%%%%%%%%%%%%%%%%%%%%%%%%%%%%%%%%%%%%%%%%%%%%%%%%%%%%%%%%%%%%%%
%%%%%%%%%%%%%%%%%%%%%%%%%%%%%%%%%%%%%%%%%%%%%%%%%%%%%%%%%%%%%%%%

\usepackage{pslatex}
\usepackage{verbatim,wrapfig}

\usepackage{amsmath,arydshln}
\usepackage{array,multirow,multicol}
\newcolumntype{R}{>{\hbox to 1.2em\bgroup\hss}{r}<{\egroup}}
\newcolumntype{T}{>{\hbox to 8em\bgroup}{c}<{\hss\egroup}}

\def\verbatimsnippet#1{\begingroup\small \verbatiminput{snippets/#1}\endgroup}
\usepackage{hyperref}

\newdimen\unitindent \unitindent=20pt
\usepackage[algo2e,noline,noend]{algorithm2e}
\newenvironment{displayalgorithm}
 {\par
  \begin{algorithm2e}[H]\leftskip=\unitindent \parskip=0pt\relax
  \DontPrintSemicolon
  \SetKwInOut{Input}{Input}\SetKwInOut{Output}{Output}
 }
 {\end{algorithm2e}\par}
\newenvironment{displayprocedure}[2]
 {\everymath{\strut}
  \begin{procedure}[H]\leftskip=\unitindent\caption{#1(#2)}}
 {\end{procedure}}
\def\sublocal{_{\mathrm\scriptstyle local}}

%%%%
%%%% Comment environment
%%%%
\usepackage{comment}
\input longshort
\specialcomment{exercise}{
  \def\CommentCutFile{exercise.cut}
  \def\PrepareCutFile{%
        \immediate\write\CommentStream{\noexpand\begin{frame}{Exercise}}}
  \def\FinalizeCutFile{%
        \immediate\write\CommentStream{\noexpand\end{frame}}}
  }{}
\excludecomment{pcse}
\excludecomment{book}

%%%%
%%%% Outlining
%%%%
\usepackage{outliner}
\OutlineLevelStart 0{\part{#1}\frame{\partpage}}    %{\frame{\part{#1}\Huge\bf #1}}
\OutlineLevelStart 1{\section{#1}} %{\frame{\section{#1}\Large\bf#1}}
\def\sectionframe#{\Level 1 }
\usepackage{framed}
\colorlet{shadecolor}{blue!15}
\OutlineLevelStart 2{\subsection{#1}
  \frame{\begin{shaded}\large #1\end{shaded}}}
%% \OutlineLevelCont 2{\end{frame}\begin{frame}{#1}}
%% \OutlineLevelEnd 2{\end{frame}}
%\OutlineTracetrue

\def\underscore{_}
\def\mpiRoutineRef#1{\RoutineRefStyle\verbatiminput{#1}}

\def\RoutineRefStyle{\scriptsize}

\newcounter{excounter}

\newenvironment
    {exerciseframe}[1][]
    {\begin{frame}[containsverbatim]
        \def\optfile{#1}
        \ifx\optfile\empty
        \frametitle{Exercise \arabic{excounter}}
        \else
        \frametitle{Exercise \arabic{excounter} (\tt{#1})}
        \fi
        \refstepcounter{excounter}}
    {\end{frame}}
\newenvironment
    {optexerciseframe}[1][]
    {\begin{frame}[containsverbatim]
        \def\optfile{#1}
        \ifx\optfile\empty
        \frametitle{Exercise (optional) \arabic{excounter}}
        \else
        \frametitle{Exercise (optional) \arabic{excounter} (\tt{#1})}
        \fi
        \refstepcounter{excounter}}
    {\end{frame}}

%\advance\textwidth by 1in
%\advance\oddsidemargin by -.5in

%%%%
%%%% Beamer customization
%%%%

\setbeamertemplate{title page}{
  \begin{picture}(0,0)
    \put(-10,10){\includegraphics[scale=.45]{tacctitle}}
    \put(50,-50){
      \begin{minipage}{10cm}
        \usebeamerfont{title}{\inserttitle\par\insertauthor\par\insertdate\par}
      \end{minipage}
    }
  \end{picture}
}

%%%%
%%%% Indexing disabled
%%%%

% \input idxmacs
%% \usepackage{makeidx}
%% \makeindex

\let\indexterm\emph
\let\indextermtt\n
\let\indextermfunction\indextermtt

\newcommand{\indextermp}[1]{\emph{#1s}}
\newcommand{\indextermsub}[2]{\emph{#1 #2}}
\newcommand{\indextermsubp}[2]{\emph{#1 #2s}}
\newcommand{\indextermbus}[2]{\emph{#1 #2}}
\newcommand{\indextermstart}[1]{\emph{#1}}
\newcommand{\indextermend}[1]{}
\newcommand{\indexstart}[1]{}
\newcommand{\indexend}[1]{}
\makeatletter
\newcommand\indexac[1]{\emph{\ac{#1}}}
\newcommand\indexacp[1]{\emph{\ac{#1}}}
\newcommand\indexacf[1]{\emph{\acf{#1}}}
\newcommand\indexacstart[1]{}
\newcommand\indexacend[1]{}
\makeatother

\iffalse
\let\indexmpishow\n
\let\indexmpidef\n
\let\indexmpiex\n
\let\indexmpi\n
\let\indexompshow\n
\let\indexompdef\n
\global\let\indexompshowdef\indexompdef
\let\indexompex\n
\let\indexomp\n
\let\mpitoindex\n
\let\mpitoindexbf\n
\let\mpitoindexit\n
\let\omptoindex\n
\let\omptoindexbf\n
\let\omptoindexit\n
%\let\indextermlet#1{\emph{#1}\index{#1|textbf}}
\let\indextermtt\n
%\let\indextermttlet\indexmpilet
\let\indexcommand\n
\let\indexclause\n
\let\indexclauselet\n
\let\indexclauseoption\n
\fi
